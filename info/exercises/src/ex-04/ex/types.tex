
% perform actual type derivation of some terms

\begin{exercise}{}

  Consider the following type system for a language with integers, conditionals, pairs, and
  functions:

  \allowdisplaybreaks
  \addtolength{\jot}{0.5em}
  \begin{gather*}
    \AxiomC{\(n\) is an integer literal}
    \UnaryInfC{\(\Gamma \vdash n : \lstinline|Int|\)}
    \DisplayProof \quad
    % variable
    \AxiomC{\(x : \tau \in \Gamma\)}
    \UnaryInfC{\(\Gamma \vdash x : \tau\)}
    \DisplayProof \\
    %
    % addition
    \AxiomC{\(\Gamma \vdash e_1 : \lstinline|Int|\)}
    \AxiomC{\(\Gamma \vdash e_2 : \lstinline|Int|\)}
    \BinaryInfC{\(\Gamma \vdash e_1 + e_2 : \lstinline|Int|\)}
    \DisplayProof \\
    %
    % multiplication
    \AxiomC{\(\Gamma \vdash e_1 : \lstinline|Int|\)}
    \AxiomC{\(\Gamma \vdash e_2 : \lstinline|Int|\)}
    \BinaryInfC{\(\Gamma \vdash e_1 \times e_2 : \lstinline|Int|\)}
    \DisplayProof \\
    %
    % booleans
    \AxiomC{\(b\) is a boolean literal}
    \UnaryInfC{\(\Gamma \vdash b : \lstinline|Bool|\)}
    \DisplayProof \quad
    \AxiomC{\(\Gamma \vdash e : \lstinline|Bool|\)}
    \UnaryInfC{\(\Gamma \vdash \lstinline|not e| : \lstinline|Bool|\)}
    \DisplayProof \\
    % boolean ops
    \AxiomC{\(\Gamma \vdash e_1 : \lstinline|Bool|\)}
    \AxiomC{\(\Gamma \vdash e_2 : \lstinline|Bool|\)}
    \BinaryInfC{\(\Gamma \vdash e_1 \land e_2 : \lstinline|Bool|\)}
    \DisplayProof
    \quad
    \AxiomC{\(\Gamma \vdash e_1 : \lstinline|Bool|\)}
    \AxiomC{\(\Gamma \vdash e_2 : \lstinline|Bool|\)}
    \BinaryInfC{\(\Gamma \vdash e_1 \lor e_2 : \lstinline|Bool|\)}
    \DisplayProof \\
    %
    % conditionals
    \AxiomC{\(\Gamma \vdash e_1 : \lstinline|Bool|\)}
    \AxiomC{\(\Gamma \vdash e_2 : \tau\)}
    \AxiomC{\(\Gamma \vdash e_3 : \tau\)}
    \TrinaryInfC{\(\Gamma \vdash if ~e_1~ then ~e_2~ else ~e_3 : \tau\)}
    \DisplayProof \\
    % pairs
    \AxiomC{\(\Gamma \vdash e_1 : \tau_1\)}
    \AxiomC{\(\Gamma \vdash e_2 : \tau_2\)}
    \BinaryInfC{\(\Gamma \vdash (e_1, e_2) : (\tau_1, \tau_2)\)}
    \DisplayProof \\
    %
    % projections
    \AxiomC{\(\Gamma \vdash e : (\tau_1, \tau_2)\)}
    \UnaryInfC{\(\Gamma \vdash fst(e) : \tau_1\)}
    \DisplayProof
    \quad
    \AxiomC{\(\Gamma \vdash e : (\tau_1, \tau_2)\)}
    \UnaryInfC{\(\Gamma \vdash snd(e) : \tau_2\)}
    \DisplayProof \\
    %
    % function
    \AxiomC{\(\Gamma \oplus \{x : \tau_1\} \vdash e : \tau_2\)}
    \UnaryInfC{\(\Gamma \vdash x \Rightarrow e : \tau_1 \to \tau_2\)}
    \DisplayProof \quad
    %
    % application
    \AxiomC{\(\Gamma \vdash e_1 : \tau_1 \to \tau_2\)}
    \AxiomC{\(\Gamma \vdash e_2 : \tau_1\)}
    \BinaryInfC{\(\Gamma \vdash e_1 e_2 : \tau_2\)}
    \DisplayProof
    %
  \end{gather*}

  \pagebreak

  \begin{enumerate}
    \item Given the following type derivation with type variables \(\tau_1,
    \ldots, \tau_5\), choose the correct options:
    \begin{equation*}
      \AxiomC{\((x, \tau_4) \in \Gamma\)}
      \UnaryInfC{\(\Gamma \vdash x: \tau_4\)}
      \UnaryInfC{\(\Gamma \vdash fst(x): \tau_3\)}
      \AxiomC{\((x, \tau_4) \in \Gamma\)}
      \UnaryInfC{\(\Gamma \vdash x: \tau_4\)}
      \UnaryInfC{\(\Gamma \vdash snd(x): \tau_5\)}
      \BinaryInfC{\(\Gamma \vdash fst(x)(snd(x)) : \tau_2\)}
      \UnaryInfC{\(\Gamma' \vdash x \Rightarrow fst(x)(snd(x)): \tau_1\)}
      \DisplayProof
    \end{equation*}

    \begin{enumerate}
      \item There are no valid assignments to the type variables such that the
      above derivation is valid.
      \item In all valid derivations, \(\tau_2 = \tau_5\).
      \item There are \emph{no} valid derivations where \(\tau_2 = \lstinline|Int|\).
      \item In all valid derivations, \(\tau_4 = (\tau_3, \tau_5)\)
      \item In all valid derivations, \(\tau_2 = \tau_4 \to \tau_1\)
      \item There is a valid derivation where \(\tau_1 = \tau_2\).
    \end{enumerate}
    \item For each of the following pairs of terms and types, provide a valid
    type derivation or briefly argue why the typing is incorrect:
    \begin{enumerate}
      \item \(x \Rightarrow x + 5\): \lstinline|Int| \(\to\) \lstinline|Int|
      \item \(x \Rightarrow y \Rightarrow x + y\): \lstinline|Int| \(\to\)
      \lstinline|Int| \(\to\) \lstinline|Int|
      \item \(x \Rightarrow y \Rightarrow y(2) \times x\): \lstinline|Int| \(\to\)
      \lstinline|Int| \(\to\) \lstinline|Int|
      \item \(x \Rightarrow (x, x)\): \lstinline|Int| \(\to\) \lstinline|(Int, Int)|
      \item \(x \Rightarrow y \Rightarrow if ~fst(x) ~then ~snd(x) ~else~ y\): \lstinline|(Bool, Int)| \(\to\) \lstinline|(Int, Int)| \(\to\) \lstinline|Int|
      \item \(x \Rightarrow y \Rightarrow if ~y~ then~ (z \Rightarrow y) ~else~ x \): (\lstinline|Bool| \(\to\) \lstinline|Bool|) \(\to\) \lstinline|Bool| \(\to\) (\lstinline|Bool| \(\to\) \lstinline|Bool|)
      \item \(x \Rightarrow y \Rightarrow if ~y~ then~ (z \Rightarrow y) ~else~ x \): (\lstinline|Int| \(\to\) \lstinline|Bool|) \(\to\) \lstinline|Bool| \(\to\) (\lstinline|Int| \(\to\) \lstinline|Bool|)
    \end{enumerate}

    \item Prove that there is \emph{no} valid type derivation for the term
    \begin{equation*}
      x \Rightarrow if~ fst(x) ~then~ snd(x)~ else~ x
    \end{equation*}
  \end{enumerate}

  \begin{solution}

    \begin{enumerate}
      \item The correct statements are d and e. For the remaining:
      \begin{itemize}
        \item \textbf{a}: set \(\tau_2 = \lstinline|Int|\), \(\tau_5 =
        \lstinline|Bool|\), \(\tau_3 = \tau_5 \to \tau_2\), \(\tau_4 = (\tau_3,
        \tau_5)\), and \(\tau_1 = \tau_4 \to \tau_2\).
        \item \textbf{b}: see (a).
        \item \textbf{c}: see (a).
        \item \textbf{f}: given \(\tau_1 = \tau_2\), we also know from the rule
        for lambda abstraction that \(\tau_1 = \tau_4 \to \tau_2\), and hence
        \(\tau_2 = \tau_4 \to \tau_2\) recursively, which is a contradiction.
      \end{itemize}

      \item For the given terms and types:
      \begin{enumerate}
        \item \(x \Rightarrow x + 5\): \lstinline|Int| \(\to\) \lstinline|Int|: \cmark
        \begin{equation*}
          \AxiomC{\(x : \lstinline|Int| \in \Gamma\)}
          \UnaryInfC{\(\Gamma \vdash x : \lstinline|Int|\)}
          \AxiomC{}
          \UnaryInfC{\(\Gamma \vdash 5 : \lstinline|Int|\)}
          \BinaryInfC{\(\Gamma \vdash x + 5 : \lstinline|Int|\)}
          \UnaryInfC{\(\Gamma' \vdash x \Rightarrow x + 5 : \lstinline|Int| \to \lstinline|Int|\)}
          \DisplayProof
        \end{equation*}
        \item \(x \Rightarrow y \Rightarrow x + y\): \lstinline|Int| \(\to\)
        \lstinline|Int| \(\to\) \lstinline|Int|: \cmark
        \item \(x \Rightarrow y \Rightarrow y(2) \times x\): \lstinline|Int|
        \(\to\) \lstinline|Int| \(\to\) \lstinline|Int|: \xmark. If \(y\) has
        type \lstinline|Int|, then \(y(2)\) cannot not well-typed, as the
        function application rule is not applicable.
        \item \(x \Rightarrow (x, x)\): \lstinline|Int| \(\to\) \lstinline|(Int, Int)|: \cmark
        \item \(x \Rightarrow y \Rightarrow if ~fst(x) ~then ~snd(x) ~else~ y\):
        \lstinline|(Bool, Int)| \(\to\) \lstinline|(Int, Int)| \(\to\)
        \lstinline|Int|: \xmark. The type of the two branches of a conditional
        must match, but here they are \lstinline|Int| and (\lstinline|Int|,
        \lstinline|Int|) respectively.
        \item \(x \Rightarrow y \Rightarrow if ~y~ then~ (z \Rightarrow y) ~else~ x \): (\lstinline|Bool| \(\to\) \lstinline|Bool|) \(\to\) \lstinline|Bool| \(\to\) (\lstinline|Bool| \(\to\) \lstinline|Bool|): \cmark
        \begin{equation*}
          \AxiomC{\((y, \lstinline|Bool|) \in \Gamma\)}
          \UnaryInfC{\(\Gamma \vdash y : \lstinline|Bool|\)}
          \AxiomC{\((x, \lstinline|Bool| \to \lstinline|Bool|) \in \Gamma\)}
          \UnaryInfC{\(\Gamma \vdash x : \lstinline|Bool| \to \lstinline|Bool|\)}
          \AxiomC{\((y, \lstinline|Bool|) \in \Gamma \oplus \{(z, \lstinline|Bool|)\}\)}
          \UnaryInfC{\(\Gamma \oplus \{(z, \lstinline|Bool|)\} \vdash y : \lstinline|Bool|\)}
          \UnaryInfC{\(\Gamma \vdash z \Rightarrow y : \lstinline|Bool| \to \lstinline|Bool|\)}
          \TrinaryInfC{\(\Gamma \vdash if ~y~ then~ (z \Rightarrow y) ~else~ x : \lstinline|Bool| \to \lstinline|Bool|\)}
          \UnaryInfC{\(\Gamma' \vdash y \Rightarrow if ~y~ then~ (z \Rightarrow y) ~else~ x : (\lstinline|Bool| \to \lstinline|Bool|) \to \lstinline|Bool| \to (\lstinline|Bool| \to \lstinline|Bool|)\)}
          \UnaryInfC{\(\Gamma'' \vdash x \Rightarrow y \Rightarrow if ~y~ then~ (z \Rightarrow y) ~else~ x : (\lstinline|Bool| \to \lstinline|Bool|) \to \lstinline|Bool| \to (\lstinline|Bool| \to \lstinline|Bool|)\)}
          \DisplayProof
        \end{equation*}
        Note that the choice of type of \(z\) (and of the argument of \(x\)) is
        arbitrary. Hence, the next typing is also valid.
        \item \(x \Rightarrow y \Rightarrow if ~y~ then~ (z \Rightarrow y) ~else~ x \): (\lstinline|Int| \(\to\) \lstinline|Bool|) \(\to\) \lstinline|Bool| \(\to\) (\lstinline|Int| \(\to\) \lstinline|Bool|): \cmark
      \end{enumerate}
      \item Non-existence of a valid type derivation for the term:
      \begin{equation*}
        t = x \Rightarrow if~ fst(x) ~then~ snd(x)~ else~ x
      \end{equation*}
      Assume that there is a valid type derivation for the term. We will attempt
      to derive a contradiction. We use the fact that if there exists a type
      derivation, every step must use one of the rules above, and that the types
      assigned to each variable must be consistent across the derivation.
      
      First, \(t\) has a type derivation \emph{if and only if} \(t_1 = if ~
      fst(x)~then~snd(x)~else~x\) has a type derivation, by using the function
      abstraction rule. We will work with \(t_1\) directly. The function
      abstraction rule here does not give us more information.

      Any type derivation for \(t_1\) must end in the conditional rule. For this
      rule to be applicable, we must have that the following are derivable:
      \begin{enumerate}
        \item \(\Gamma \vdash fst(x) : \lstinline|Bool|\)
        \item \(\Gamma \vdash snd(x) : \tau\)
        \item \(\Gamma \vdash x : \tau\)
      \end{enumerate}
      where the type variable \(\tau\) is also the type of \(t_1\).

      By using the projection rule on (a) and (b), we learn that the type of
      \(x\) must be \((\lstinline|Bool|, \tau_1)\) and \((\tau_2, \tau)\) for
      two fresh variables \(\tau_1\) and \(\tau_2\) respectively. Matching the
      two, as \(x\) may only have one type, we must have \(\tau_1 = \tau\),
      \(\tau_2 = \lstinline|Bool|\), and thus the type of \(x\) is
      \((\lstinline|Bool|, \tau)\).

      However, from (c), we learn that the type of \(x\) is \(\tau\). It must be
      the case that \(\tau = (\lstinline|Bool|, \tau)\). This is not possible
      for any type \(\tau\), and we have a contradiction.

      Hence, there is no valid type derivation for the term \(t\).
    \end{enumerate}
    
  \end{solution}

\end{exercise}
