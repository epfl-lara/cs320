
\begin{exercise}{}
  For each of the following pairs of grammars, show that they are equivalent by
  identifying them with inductive relations, and proving that the inductive
  relations contain the same elements.

  \begin{enumerate}
    \item 
    \(A_1 : S ::= S + S \mid \num \) \\ 
    \(A_2 : R ::= \num ~R' \text{ and } R' ::= + R~ R' \mid \epsilon\)
    \item 
    \(B_1 : S ::= S(S)S \mid \epsilon \) \\ 
    \(B_2 : R ::= RR \mid (R) \mid \epsilon\)
  \end{enumerate}

  \begin{solution}
    \begin{enumerate}
      \item \(A_2\) is the result of left-recursion elimination on \(A_1\).
      First, expressing them as inductive relations, with rules named as on the
      right:
      %
      \addtolength{\jot}{1ex}
      \begin{gather*}
        \AxiomC{\phantom{\(w_1 \in S\)}}
        \RightLabel{\(S_{num}\)}
        \UnaryInfC{\(\num \in S\)}
        \DisplayProof 
        \quad
        \AxiomC{\(w_1 \in S\)}
        \AxiomC{\(w_2 \in S\)}
        \RightLabel{\(S_+\)}
        \BinaryInfC{\(w_1 + w_2 \in S\)}
        \DisplayProof \\
        %
        \AxiomC{\(w \in S\)}
        \RightLabel{\(A_{1}^{start}\)}
        \UnaryInfC{\(w \in A_1\)}
        \DisplayProof \\
        %
        \AxiomC{\(w \in R'\)}
        \RightLabel{\(R_{num}\)}
        \UnaryInfC{\(\num~ w\in R\)}
        \DisplayProof \\
        %
        \AxiomC{\(w \in R\)}
        \AxiomC{\(w' \in R'\)}
        \RightLabel{\(R'_{+}\)}
        \BinaryInfC{\(+w ~ w' \in R'\)}
        \DisplayProof 
        \quad
        \AxiomC{\phantom{\(w' \in R'\)}}
        \RightLabel{\(R'_{\epsilon}\)}
        \UnaryInfC{\(\epsilon \in R'\)}
        \DisplayProof \\
        %
        \AxiomC{\(w \in R\)}
        \RightLabel{\(A_{2}^{start}\)}
        \UnaryInfC{\(w \in A_2\)}
        \DisplayProof
      \end{gather*}

      We must show that for any word \(w\), \(w \in A_1\) if and only if \(w \in
      A_2\). For this, it must be the case that there is a derivation tree for
      \(w \in A_1\) (equivalently, \(w \in S\)) if and only if there is a
      derivation tree for \(w \in A_2\) (equivalently, \(w \in R\)) according to
      the inference rules above.
      \begin{enumerate}
        \item \(w \in S \implies w \in R\): we induct on the depth of the
        derivation tree. 
        \begin{itemize}
          \item Base case: derivation tree of depth 1. The tree must be
          \begin{gather*}
            \AxiomC{}
            \RightLabel{\(S_{num}\)}
            \UnaryInfC{\(\num \in S\)}
            \DisplayProof
          \end{gather*}
          We can show that there is a corresponding derivation tree for \(w \in R\):
          \begin{gather*}
            \AxiomC{}
            \RightLabel{\(R'_{\epsilon}\)}
            \UnaryInfC{\(\epsilon \in R'\)}
            \RightLabel{\(R_{num}\)}
            \UnaryInfC{\(\num \in R\)}
            \DisplayProof
          \end{gather*}
          \item Inductive case: derivation tree of depth \(n+1\), given that for
          every derivation of depth \( \le n\) of \(w' \in S\) for any \(w'\), there
          is a corresponding derivation of \(w' \in R\). The last rule applied
          in the derivation must be \(S_+\):
          \begin{gather*}
            \AxiomC{\ldots}
            \UnaryInfC{\(w_1 \in S\)}
            \AxiomC{\ldots}
            \UnaryInfC{\(w_2 \in S\)}
            \RightLabel{\(S_+\)}
            \BinaryInfC{\(w_1 + w_2 \in S\)}
            \DisplayProof
          \end{gather*}
          By the inductive hypothesis, since \(w_1 \in S\) and \(w_2 \in S\)
          have a derivation tree of smaller depth, there are derivation trees
          for \(w_1 \in R\) and \(w_2 \in R\). In particular, the derivation for
          \(w_1 \in R\) must end with the rule \(R_{num}\) (only case), so there
          must be a derivation tree for \(\num ~w_1' \in R\) with \(w_1' \in
          R'\) and \(num~w_1' = w_1\). We have the following pieces:
          \begin{gather*}
            \AxiomC{\ldots}
            \UnaryInfC{\(w_1' \in R'\)}
            \RightLabel{\(R_{num}\)}
            \UnaryInfC{\(\num ~w_1' \in R\)}
            \DisplayProof
            \quad
            \AxiomC{\ldots}
            \UnaryInfC{\(w_2 \in R\)}
            \DisplayProof
          \end{gather*}
          To show that \(w_1 + w_2 \in R\), i.e. \(\num ~w_1' + w_2 \in R\), we
          must first show that \(w_1' + w_2 \in R'\), as required by the rule
          \(R_{num}\). Note that words in \(R'\) are of the form \((+ \num)^*\).
          We will prove this separately for all pairs of words at the end
          (\(R'_{Lemma}\)). Knowing this, however, we can construct the
          derivation tree for \(w_1 + w_2 \in R\):
          \begin{gather*}
            \AxiomC{\ldots}
            \UnaryInfC{\(w_1' \in R'\)}
            \AxiomC{\ldots}
            \UnaryInfC{\(w_2 \in R\)}
            \RightLabel{\(R'_{Lemma}\)}
            \BinaryInfC{\(w_1' + w_2 \in R'\)}
            \RightLabel{\(R_{num}\)}
            \UnaryInfC{\(\num ~w_1' + w_2 \in R\)}
            \DisplayProof
          \end{gather*}
          \(\num ~w_1' + w_2 = w_1 + w_2 = w\), as required.

          Finally, we will show the required lemma. We will prove a stronger
          property \(R'_{concat}\) first, that for any pair of words \(w_1, w_2
          \in R'\), \(w_1 ~w_2 \in R'\) as well. We induct on the derivation of
          \(w_1 \in R'\).
          
          Base case: derivation ends with \(R'_\epsilon\). Then \(w_1 =
          \epsilon\), and \(w_1 ~w_2 = w_2 \in R'\) by assumption. 
          
          Inductive case: derivation ends with \(R'_+\). Then \(w_1 = + v v'\)
          for some \(v \in R\) and \(v' \in R'\):
          \begin{gather*}
            \AxiomC{\ldots}
            \UnaryInfC{\(v \in R\)}
            \AxiomC{\ldots}
            \UnaryInfC{\(v' \in R'\)}
            \RightLabel{\(R'_+\)}
            \BinaryInfC{\(+ v ~v' \in R'\)}
            \DisplayProof
          \end{gather*}

          Since \(v' \in R'\) has a smaller derivation tree than \(w_1\), by the
          inductive hypothesis, we can prove that \(v'~w_2 \in R'\). We get:
          \begin{gather*}
            \AxiomC{\ldots}
            \UnaryInfC{\(v \in R\)}
            \AxiomC{\ldots}
            \UnaryInfC{\(v' \in R'\)}
            \AxiomC{\ldots}
            \UnaryInfC{\(w_2 \in R'\)}
            \RightLabel{\(R'_{concat}\)}
            \BinaryInfC{\(v' ~w_2 \in R'\)}
            \RightLabel{\(R'_+\)}
            \BinaryInfC{\(+ v ~v' ~w_2 \in R'\)}
            \DisplayProof
          \end{gather*}

          So, \(R'_{concat}\) is proven. We can show \(R'_{lemma}\), i.e. \(w_1'
          + w_2 \in R'\) if \(w_1' \in R'\) and \(w_2 \in R\) as:
          \begin{gather*}
            \AxiomC{\ldots}
            \UnaryInfC{\(w_1' \in R'\)}
            \AxiomC{\ldots}
            \UnaryInfC{\(w_2 \in R\)}
            \AxiomC{}
            \RightLabel{\(R'_\epsilon\)}
            \UnaryInfC{\(\epsilon \in R'\)}
            \RightLabel{\(R'_+\)}
            \BinaryInfC{\(+ w_2 \in R'\)}
            \RightLabel{\(R'_{concat}\)}
            \BinaryInfC{\(w_1' + w_2 \in R'\)}
            \DisplayProof
          \end{gather*}

          Thus, the proof is complete.
      \end{itemize}
      \item \(w \in R \implies w \in S\): we induct on the depth of the
      derivation tree for \(w \in R\). This direction is simpler than the other,
      but the general method is similar.
      
      \begin{itemize}
        \item Base case: derivation tree of depth 2 (minimum). The tree must be
        \begin{gather*}
          \AxiomC{}
          \RightLabel{\(R'_{\epsilon}\)}
          \UnaryInfC{\(\epsilon \in R'\)}
          \RightLabel{\(R_{num}\)}
          \UnaryInfC{\(\num \in R\)}
          \DisplayProof
        \end{gather*}
  
        We have the corresponding derivation tree for \(w \in S\):
        \begin{gather*}
          \AxiomC{}
          \RightLabel{\(S_{num}\)}
          \UnaryInfC{\(\num \in S\)}
          \DisplayProof
        \end{gather*}
  
        \item Inductive case: derivation tree of depth \(n+1\), given that for
        every derivation of depth \(\le n\) of \(w' \in R\) for any \(w'\),
        there is a corresponding derivation of \(w' \in S\). The last rules
        applied must be \(R_{num}\) and \(R'_{+}\) (otherwise the derivation
        would be of the base case):
        \begin{gather*}
          \AxiomC{\ldots}
          \UnaryInfC{\(w_1 \in R\)}
          \AxiomC{\ldots}
          \UnaryInfC{\(w_2 \in R'\)}
          \RightLabel{\(R'_{+}\)}
          \BinaryInfC{\(+ w_1 ~w_2 \in R'\)}
          \RightLabel{\(R_{num}\)}
          \UnaryInfC{\(\num + w_1~ w_2 \in R\)}
          \DisplayProof
        \end{gather*}
        %
        where \(w = \num + w_1 ~w_2\). However, we are somewhat stuck here, as
        we have no way to relate \(R'\) and \(S\). We will separately show that
        if \(+w' \in R'\), then there is a derivation of \(w' in S\) (lemma
        \(R'_{S}\)). This will allow us to complete the proof:
        \begin{gather*}
          \AxiomC{}
          \RightLabel{\(S_{num}\)}
          \UnaryInfC{\(\num \in S\)}
          \AxiomC{\ldots}
          \UnaryInfC{\(+w_1 ~w_2 \in R'\)}
          \RightLabel{\(R'_{S}\)}
          \UnaryInfC{\(w_1 ~w_2 \in S\)}
          \RightLabel{\(S_{+}\)}
          \BinaryInfC{\(\num + w_1 ~w_2 \in S\)}
          \DisplayProof
        \end{gather*}

        The proof of the lemma \(R'_S\) is by induction again, and not shown
        here. This completes the original proof.
      \end{itemize}
        
      \end{enumerate}
      \item Argument similar to Exercise Set 2 Problem 4 (same pair of
      grammars). \(B_1 \subseteq B_2\) as relations can be seen by producing a
      derivation tree for each possible case in \(B_1\). For the other
      direction, \(B_2 \subseteq B_1\), it is first convenient to prove
      that \(B_1\) is closed under concatenation, i.e., if \(w_1, w_2 \in B_1\)
      then there is a derivation tree for \(w_1 ~ w_2 \in B_1\).
    \end{enumerate}
  \end{solution}
  
\end{exercise}

