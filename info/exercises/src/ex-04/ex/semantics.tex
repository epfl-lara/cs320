

\begin{exercise}{}

  Consider the following expression language over naturals, and a \emph{halving}
  operator:
  %
  \begin{align*}
    expr ::= \half(expr) \mid expr + expr \mid \num
  \end{align*}
  where \(\num\) is any natural number constant \(\ge 0\).

  We will design the operational semantics of this language. The semantics
  should define rules that apply to as many expressions as possible, while being
  subjected to the following safety conditions:
  \begin{itemize}
    \item the semantics should \emph{not} permit halving unless the argument is even
    \item they should evaluate operands from left-to-right
  \end{itemize}

  Of the given rules below, choose a \emph{minimal} set that satisfies the
  conditions above. A set is \emph{not} minimal if removing any rule does not
  change the set of expressions that can be evaluated by the semantics, i.e. the
  domain of \(\leadsto, \{x \mid \exists y.\; x \leadsto y\}\), remains
  unchanged. The removed rule is said to be \emph{redundant}.

  
  % \begin{multicols}{2}
    \allowdisplaybreaks
    \addtolength{\jot}{2ex}
    \begin{gather*}
      \AxiomC{\(e \leadsto e'\)}
      \UnaryInfC{\(\half(e) \leadsto e'\)}
      \DisplayProof \tag{A} \\
      %
      \AxiomC{\(n\) is a value}
      \AxiomC{\(n = 2k\)}
      \BinaryInfC{\(\half(n) \leadsto k\)}
      \DisplayProof \tag{B} \\
      %
      \AxiomC{\(n\) is a value}
      \UnaryInfC{\(\half(n) \leadsto \lfloor \frac{n}{2} \rfloor\)}
      \DisplayProof \tag{C} \\
      %
      \AxiomC{\(\half(e) \leadsto \half(e')\)}
      \UnaryInfC{\(\half(e) \leadsto e'\)}
      \DisplayProof \tag{D} \\
      %
      \AxiomC{\(e \leadsto e'\)}
      \UnaryInfC{\(\half(e) \leadsto \half(e')\)}
      \DisplayProof \tag{E} \\
      %
      \AxiomC{\(e' \leadsto \half(e)\)}
      \UnaryInfC{\(half(e) \leadsto e'\)}
      \DisplayProof \tag{F} \\
      %
      \AxiomC{\(n_1\) is a value}
      \AxiomC{\(n_2\) is a value}
      \AxiomC{\(n_1 + n_2 = k\)}
      \AxiomC{\(n_1\) is odd}
      \QuaternaryInfC{\(n_1 + n_2 \leadsto k\)}
      \DisplayProof \tag{G} \\
      %
      \AxiomC{\(e \leadsto e'\)}
      \AxiomC{\(n\) is a value}
      \BinaryInfC{\(n + e \leadsto n + e'\)}
      \DisplayProof \tag{H} \\
      %
      \AxiomC{\(e_2 \leadsto e_2'\)}
      \UnaryInfC{\(e_1 + e_2 \leadsto e_1 + e_2'\)}
      \DisplayProof \tag{I} \\
      %
      \AxiomC{\(n_1\) is a value}
      \AxiomC{\(n_2\) is a value}
      \AxiomC{\(n_1 + n_2 = k\)}
      \AxiomC{\(n_1, n_2\) are even}
      \QuaternaryInfC{\(n_1 + n_2 \leadsto k\)}
      \DisplayProof \tag{J} \\
      %
      \AxiomC{\(n_1\) is a value}
      \AxiomC{\(n_2\) is a value}
      \AxiomC{\(n_1 + n_2 = k\)}
      \TrinaryInfC{\(n_1 + n_2 \leadsto k\)}
      \DisplayProof \tag{K} \\
      %
      \AxiomC{\(e_1 \leadsto e_1'\)}
      \UnaryInfC{\(e_1 + e_2 \leadsto e_1' + e_2\)}
      \DisplayProof \tag{L} 
    \end{gather*}
  % \end{multicols}

  \begin{solution}
    A possible such minimal set of rules is \(\{B, E, H, K, L\}\). % TODO:

    On what happens when the other rules are added to this set:
    \begin{itemize}
      \item A: incorrect; allows deducing \(\half(\half(10)) \leadsto 5\) with
      rule B.
      \item C: incorrect; allows deducing \(\half(3) \leadsto 1\).
      \item D: incorrect; allows deducing \(\half(\half(10)) \leadsto 5\) with
      rules B and E.
      \item F: redundant; reverses a reduction.
      \item G: redundant; special case of rule K.
      \item I: incorrect; does not reduce the expression left-to-right.
      \item J: redundant; special case of rule K.
    \end{itemize}
  \end{solution}

\end{exercise}

\begin{exercise}{}

  Consider a simple programming language with integer arithmetic, boolean
  expressions, and user-defined functions:

  \begin{align*}
    expr ::= &~ true \mid false \mid \num \\
             &~ expr == expr \mid expr + expr \\
             &~ expr ~\&\& ~expr \mid if ~(expr) ~expr ~else ~expr \\
             &~ f(expr, \ldots, expr) \mid x
  \end{align*}
  %
  where \(f\) represents a (user-defined) function, \(x\) represents a
  variable, and \(\num\) represents an integer.

  \begin{enumerate}
    \item Inductively define a substitution operation for the terms in this
      language, which replaces every free occurrence of a variable \(x\) with a
      given expression \(e\).

      The rule for substitution in an addition is provided as an example. Here,
      \(t[x := e]\) represents the term \(t\), with every free occurrence of \(x\)
      simultaneously replaced by \(e\).

      \begin{equation*}
        \AxiomC{\(t_1[x := e] \to t_1'\)}
        \AxiomC{\(t_2[x := e] \to t_2'\)}
        \BinaryInfC{\(t_1 + t_2[x := e] \to t_1' + t_2'\)}
        \DisplayProof
      \end{equation*}

    \item Write the rules for the operational semantics for this language,
    assuming \emph{call-by-name} semantics for function calls. In call-by-name
    semantics, function arguments are not evaluated before the call. Instead,
    the parameters are merely substituted into the function body. You may assume
    that function parameters are named distinctly from variables in the program.

    \item Under the following environment (with function names, parameters, and
    bodies):
      \begin{gather*}
        (sum, [x], if ~ (x == 0) ~ then ~ 0 ~ else ~ x + sum(x + (-1))) \\
        (rec, [~], rec()) \\
        (default, [b, x], if ~ b ~ then ~ x ~ else ~ 0)
      \end{gather*}

      evaluate each of the following expressions, showing the derivations:
      \begin{enumerate}
        \item \(sum(2)\)
        \item \(if ~(1 == 2) ~ then ~ 3 ~ else ~ 4\)
        \item \(sum(sum(0))\)
        \item \(rec()\)
        \item \(default(false, rec())\)
      \end{enumerate}

      How would the evaluations in each case change if we used \emph{call-by-value}
      semantics instead?
  \end{enumerate}

  \begin{solution}
    \allowdisplaybreaks

    \begin{enumerate}
      \item Substitution rules:
      \begin{gather*}
        \AxiomC{}
        \UnaryInfC{\(true[x := e] \to true\)}
        \DisplayProof \\
        %
        \AxiomC{}
        \UnaryInfC{\(false[x := e] \to false\)}
        \DisplayProof \\
        %
        \AxiomC{}
        \UnaryInfC{\(\num[x := e] \to \num\)}
        \DisplayProof \\
        %
        \AxiomC{\(t_1[x := e] \to t_1'\)}
        \AxiomC{\(t_2[x := e] \to t_2'\)}
        \BinaryInfC{\(t_1 == t_2[x := e] \to t_1' == t_2'\)}
        \DisplayProof \\
        %
        \AxiomC{\(t_1[x := e] \to t_1'\)}
        \AxiomC{\(t_2[x := e] \to t_2'\)}
        \BinaryInfC{\(t_1 + t_2[x := e] \to t_1' + t_2'\)}
        \DisplayProof \\
        %
        \AxiomC{\(t_1[x := e] \to t_1'\)}
        \AxiomC{\(t_2[x := e] \to t_2'\)}
        \BinaryInfC{\(t_1 ~\&\& ~t_2[x := e] \to t_1' ~\&\& ~t_2'\)}
        \DisplayProof \\
        %
        \AxiomC{\(t_1[x := e] \to t_1'\)}
        \AxiomC{\(t_2[x := e] \to t_2'\)}
        \AxiomC{\(t_3[x := e] \to t_3'\)}
        \TrinaryInfC{\(if ~(t_1) ~t_2 ~else ~t_3[x := e] \to if ~(t_1') ~t_2' ~else ~t_3'\)}
        \DisplayProof \\
        %
        \AxiomC{\(t_1[x := e] \to t_1'\)}
        \AxiomC{\(\ldots\)}
        \AxiomC{\(t_n[x := e] \to t_3'\)}
        \TrinaryInfC{\(f(t_1, \ldots, t_n)[x := e] \to f(t_1', \ldots, t_n')\)}
        \DisplayProof \\
        %
        \AxiomC{}
        \UnaryInfC{\(x[x := e] \to e\)}
        \DisplayProof \\
        %
        \AxiomC{\(x \neq y\)}
        \UnaryInfC{\(y[x := e] \to y\)}
        \DisplayProof
      \end{gather*}

      \item Operational semantics:
      \begin{itemize}
        \item Equality:
          \begin{gather*}
            \AxiomC{\(n_1, n_2\) are integer values}
            \AxiomC{\(n_1 = n_2\)}
            \BinaryInfC{\(n_1 == n_2 \leadsto true\)}
            \DisplayProof \\
            %
            \AxiomC{\(n_1, n_2\) are integer values}
            \AxiomC{\(n_1 \neq n_2\)}
            \BinaryInfC{\(n_1 == n_2 \leadsto false\)}
            \DisplayProof
          \end{gather*}
        \item Addition:
          \begin{gather*}
            \AxiomC{\(t_1 \leadsto t_1'\)}
            \UnaryInfC{\(t_1 + t_2 \leadsto t_1' + t_2\)}
            \DisplayProof \\
            %
            \AxiomC{\(n\) is an integer value}
            \AxiomC{\(t_2 \leadsto t_2'\)}
            \BinaryInfC{\(n + t_2 \leadsto n + t_2'\)}
            \DisplayProof \\
            %
            \AxiomC{\(n_1, n_2\) are integer values}
            \AxiomC{\(n_1 + n_2 = k\)}
            \BinaryInfC{\(n_1 + n_2 \leadsto k\)}
            \DisplayProof
          \end{gather*}
        \item Conjunction:
          \begin{gather*}
            \AxiomC{\(t_1 \leadsto t_1'\)}
            \UnaryInfC{\(t_1 ~\&\& ~t_2 \leadsto t_1' ~\&\& ~t_2\)}
            \DisplayProof \\
            %
            \AxiomC{}
            \UnaryInfC{\(true ~\&\&~ t \leadsto t\)}
            \DisplayProof \\
            %
            \AxiomC{}
            \UnaryInfC{\(false ~\&\&~ t \leadsto false\)}
            \DisplayProof
          \end{gather*}
        \item Conditionals:
          \begin{gather*}
            \AxiomC{\(t_1 \leadsto t_1'\)}
            \UnaryInfC{\(if ~(t_1) ~t_2 ~else ~t_3 \leadsto if ~(t_1') ~t_2 ~else ~t_3\)}
            \DisplayProof \\
            %
            \AxiomC{}
            \UnaryInfC{\(if ~(true) ~t_2 ~else ~t_3 \leadsto t_2\)}
            \DisplayProof \\
            %
            \AxiomC{}
            \UnaryInfC{\(if ~(false) ~t_2 ~else ~t_3 \leadsto t_3\)}
            \DisplayProof
          \end{gather*}
        \item Function call:
          \begin{gather*}
            \alwaysNoLine
            \AxiomC{\(b_0\) is the body of \(f\)}
            \UnaryInfC{\((x_1, \ldots, x_n)\) are parameters of \(f\)}
            \UnaryInfC{\(b_0[x_1 := t_1] \to b_1 \quad \ldots \quad b_{n-1}[x_n := t_n] \to b_n\)}
            \alwaysSingleLine
            \UnaryInfC{\(f(t_1, \ldots, t_n) \leadsto b_n\)}
            \DisplayProof
          \end{gather*}
      \end{itemize}

      \item Evaluations:
      \begin{enumerate}
        \item \(sum(2) \leadsto 3\):
          \begin{align*}
            &sum(2)\\
            &\leadsto if ~(2 == 0) ~ then ~ 0 ~ else ~ 2 + sum(2 + (-1))\\
            &\leadsto if ~(false) ~ then ~ 0 ~ else ~ 2 + sum(2 + (-1))\\
            &\leadsto 2 + sum(2 + (-1))\\
            &\leadsto 2 + if ~((2 + (-1)) == 0) ~ then ~ 0 ~ else ~ 1 + sum(2 + (-1))\\
            &\leadsto 2 + if ~(1 == 0) ~ then ~ 0 ~ else ~ 1 + sum(2 + (-1))\\
            &\leadsto 2 + if ~(false) ~ then ~ 0 ~ else ~ 1 + sum(2 + (-1) + (-1))\\
            &\leadsto 2 + (1 + sum(2 + (-1) + (-1)))\\
            & \ldots \\
            &\leadsto 2 + (1 + 0) \\
            &\leadsto 2 + 1 \\
            &\leadsto 3
          \end{align*}
        \item \(sum(sum(0)) \leadsto 0\):
        \begin{align*}
          &sum(sum(0))\\
          &\leadsto if ~(sum(0) == 0) ~ then ~ 0 ~ else ~ 0 + sum(sum(0) + (-1))\\
          &\leadsto \ldots \textit{(expand \(sum(0)\) in the conditional)} \\
          &\leadsto if ~(0 == 0) ~ then ~ 0 ~ else ~ 0 + sum(sum(0) + (-1))\\
          &\leadsto if ~(true) ~ then ~ 0 ~ else ~ 0 + sum(sum(0) + (-1))\\
          &\leadsto 0
        \end{align*}
        \item \(if ~(1 == 2) ~ then ~ 3 ~ else ~ 4 \leadsto 4\).
        \item \(rec() \leadsto rec()\) (infinite loop).
        \item \(default(false, rec()) \leadsto 0\).
      \end{enumerate}

      Under call-by-value-semantics, the structure of the evaluations would be
      different. In \(sum(sum(0))\), we would evaluate the inner \(sum(0)\) to
      \(0\) before evaluating the outer \(sum(\cdot)\). In \(default(false,
      rec())\), we would need to evaluate \(rec()\), which would lead to an
      infinite loop.
    \end{enumerate}
        
  \end{solution}
  
\end{exercise}
