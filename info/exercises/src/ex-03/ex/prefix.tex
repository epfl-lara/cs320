
\begin{exercise}{}

  If \(L\) is a regular language, then the set of prefixes of words in \(L\) is
  also a regular language. Given this fact, from a regular expression for \(L\),
  we should be able to obtain a regular expression for the set of all prefixes
  of words in \(L\) as well.
  
  We want to do this with a function \(\prefixes\) that is recursive over the
  structure of the regular expression for \(L\), i.e. of the form:
  %
  \begin{align*}
    \prefixes(\epsilon) &= \epsilon \\
    \prefixes(a) &= a \mid \epsilon \\
    \prefixes(r \mid s) &= \prefixes(r) \mid \prefixes(s) \\
    \prefixes(r \cdot s) &= \ldots \\
    \prefixes(r^*) &= \ldots \\
    \prefixes(r^+) &= \ldots
  \end{align*}

  \begin{enumerate}
    \item Complete the definition of \(\prefixes\) above by filling in the
    missing cases.
    \item Use this definition to find:
    \begin{enumerate}
      \item \(\prefixes(ab^*c)\)
      \item \(\prefixes((a \mid bc)^*)\)
    \end{enumerate}
  \end{enumerate}

  \begin{solution}
    The computation for \(\prefixes(\cdot)\) is similar to the computation of
    \(\first(\cdot)\) for grammars.

    \begin{enumerate}
      \item The missing cases:
      \begin{enumerate}
        \item \(\prefixes(r \cdot s) = \prefixes(r) \mid r \cdot \prefixes(s)\).
        Either we have read \(r\) partially, or we have read all of \(r\), and a
        part of \(s\).
        \item \(\prefixes(r^*) = r*\cdot\prefixes(r)\). We can
        consider \(r^* = \epsilon \mid r \mid rr \mid \ldots\), and apply the
        rules for union and concatenation. Intuitively, if the word has \(n \ge
        0\) instances of \(r\), we can read \(m < n\) instances of \(r\), and
        then a prefix of the next instance of \(r\).
        \item \(\prefixes(r^+) = r^* \cdot \prefixes(r)\). Same as
        previous. Why does the empty case still appear?
      \end{enumerate}
      \item The prefix computations are:
      \begin{enumerate}
        \item \(\prefixes(ab^*c) = \epsilon \mid a \mid ab^*(b \mid c \mid \epsilon)\). Computation:
        \begin{align*}
          \prefixes(ab^*c) &= \prefixes(a) \mid a\cdot\prefixes(b^*c) & [\text{concatenation}]\\
                           &= (a \mid \epsilon) \mid a\cdot\prefixes(b^*c) &[a]\\
                           &= (a \mid \epsilon) \mid a\cdot(\prefixes(b^*) \mid b^*\prefixes(c)) &[\text{concatenation}]\\
                           &= (a \mid \epsilon) \mid a\cdot(\prefixes(b^*) \mid b^*(c \mid \epsilon)) &[c]\\
                           &= (a \mid \epsilon) \mid a\cdot(b^*\prefixes(b) \mid b^*(c \mid \epsilon)) &[\text{star}]\\
                           &= (a \mid \epsilon) \mid a\cdot(b^*(b \mid \epsilon) \mid b^*(c \mid \epsilon)) &[b]\\
                           &= (a \mid \epsilon) \mid a\cdot(b^*(b \mid c \mid \epsilon)) &[\text{rewrite}]\\
                           &= \epsilon \mid a \mid a\cdot(b^*(b \mid c \mid \epsilon)) & [\text{rewrite}]\\
        \end{align*}
        \item \(\prefixes((a \mid bc)^*) = (a \mid bc)^*(\epsilon \mid a \mid b \mid bc)\).
      \end{enumerate}
    \end{enumerate}
  \end{solution}
  
\end{exercise}
