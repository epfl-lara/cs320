
% this language is not LL 1 actually, I think

% \begin{exercise}{}
%   Consider the following grammar of \(\mathbf{if}-\mathbf{then}-\mathbf{else}\) expressions with assignments:
%   %
%   \begin{align*}
%     stmt &::= \mathbf{if} ~id = id~ \mathbf{then} ~stmt ~optStmt \\
%          &::= \{ stmt^* \} \\
%          &::= id = id; \\
%     optStmt &::= \epsilon \mid \mathbf{else} ~stmt \\
%   \end{align*}
  
%   \begin{enumerate}
%     \item Show that the grammar is ambiguous.
%     \item Is the grammar LL(1)?
%   \end{enumerate}

% \end{exercise}


\begin{exercise}{}
  Argue that the following grammar is \emph{not} LL(1). Produce an equivalent
  LL(1) grammar.

  \begin{equation*}
    E ::= \mathbf{num} + E \mid \mathbf{num} - E \mid \mathbf{num}
  \end{equation*}

  \begin{solution}
    The language is clearly not LL(1), as on seeing a token \(\mathbf{num}\), we
    cannot decide whether to continue parsing it as \(\mathbf{num} + E\),
    \(\mathbf{num} - E\), or the end. 

    The notable problem is the common prefix between the rules. We can separate
    this out by introducing a new non-terminal \(T\). This is a transformation
    known as \emph{left factorization}.

    \begin{align*}
      E &::= \mathbf{num} ~T \\
      T &::= + E \mid - E \mid \epsilon
    \end{align*}

    % without changing the terms or the overall "structure" of the grammar, we
    % have logically partitioned it to fit within our parsing schema.

  \end{solution}
\end{exercise}

\begin{exercise}{}
  Consider the following grammar:

  \begin{equation*}
    S ::= S(S) \mid S[S] \mid () \mid [\;]
  \end{equation*}
    
  Check whether the same transformation as the previous case can be applied to
  produce an LL(1) grammar. If not, argue why, and suggest a different
  transformation.

  \begin{solution}
    Applying left factorization to the grammar, we get:
    \begin{align*}
      S &::= S ~T \mid S ~T \mid () \mid [\;] \\
      T &::= (S) \mid [S]
    \end{align*}

    This is not LL(1), as on reading a token ``\((\)'', we cannot decide whether
    this is the final parentheses (base case) in the expression, or whether
    there is a \(T\) following it.

    The problem is that this version of the grammar is left-recursive. A
    recursive-descent parser for this grammar would loop forever on the first
    rule. This is caused by the fact that our parsers are top-down, left to
    right. We can fix this by \emph{moving} the recursion to the right. This is
    generally called \emph{left recursion elimination}.

    Transformed grammar steps (explanation below):
    \begin{align*}
      S &::= ()S' \mid [\;]S' \\
      S' &::= (S)S' \mid [S]S' \mid \epsilon
    \end{align*}

    To eliminate left-recursion in general, consider a non-terminal \(A ::=
    A\alpha \mid \beta\), where \(\beta\) does not start with \(A\) (not
    left-recursive). We can remove the left recursion by introducing a new
    non-terminal, \(A'\), such that:
    \begin{align*}
      A &::= A' \mid \beta A' \\
      A' &::= \alpha A' \mid \epsilon
    \end{align*}
    i.e., for the left-recursive rule \(A\alpha\), we instead attempt to parse
    an \(\alpha\) followed by the rest. In exchange, the base case \(\beta\) now
    expects an \(A'\) to follow it.
    %
    Note that \(\beta\) can be empty as well.
    
    Intuitively, we are shifting the direction in which we look for instances of
    \(A\). Consider a partial derivation starting from \(\beta \alpha \alpha
    \alpha\). The original version of the grammar would complete the parsing as:
    \begin{center}
      \begin{forest}
        [\(A\)
          [\(A\)
            [\(A\)
              [\(A\)
                [\(\beta\)]
              ]
              [\(\alpha\)]
            ]
            [\(\alpha\)]
          ]
          [\(\alpha\)]
        ]     
      \end{forest}
    \end{center}
    but with the new grammar, we parse it as:
    \begin{center}
      \begin{forest}
        [\(A\)
          [\(\beta\)]
          [\(A'\)
            [\(\alpha\)]
            [\(A'\)
              [\(\alpha\)]
              [\(A'\)
                [\(\alpha\)]
                [\(A'\)
                  [\(\epsilon\)]
                ]
              ]
            ]
          ]
        ]
      \end{forest}
    \end{center}

    There are two main pitfalls to remember with left-recursion elimination:
    \begin{enumerate}
      \item it may need to be applied several times till the grammar is
      unchanged, as the first transformation may introduce new (indirect)
      recursive rules (check \(A ::= AA\alpha \mid \epsilon\)).
      \item it may require \emph{inlining} some non-terminals, when the left
      recursion is \emph{indirect}. For example, consider \(A ::= B\alpha, B ::=
      A\beta\), where there is no immediate reduction to do, but inlining \(B\),
      we get \(A ::= A\beta\alpha\), where the elimination can be applied.
    \end{enumerate}
  \end{solution}

\end{exercise}
