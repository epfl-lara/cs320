
% do type checking for some simple terms

\pagebreak
\begin{exercise}{}
  For each of the following term-type pairs \((t, \tau)\), check whether the
  term can be ascribed with the given type, i.e., whether there exists a
  derivation of \(\Gamma \vdash t: \tau\) for some typing context \(\Gamma\) in
  the given system. If not, briefly argue why.

  \begin{enumerate}
    \item \(\lstinline|x|, \abool\)
    \item \(\lstinline|x| + 1, \aint\)
    \item \(\lstinline|(x && y)  ==  (x <=  0)|, \abool\)
    \item \(\lstinline|f| ~\ato~ \lstinline|x| ~\ato~ \lstinline|y| ~\ato~ \lstinline|f((x, y))|\): 
    
    \lstinline|((List[Int], Bool) => Int) => List[Int] => Bool => Int|
    \item \lstinline|Cons(x, x)| : \lstinline|List[List[Int]]|
  \end{enumerate}

  \begin{solution}
    \begin{enumerate}
      \item \lstinline|x, Bool|. Derivation, assume \lstinline|x| is a boolean:
      \begin{equation*}
        \AxiomC{\((\lstinline|x|, \abool) \in \{(\lstinline|x|, \abool)\}\)}
        \UnaryInfC{\(\{(\lstinline|x|, \abool)\} \vdash \lstinline|x: Bool|\)}
        \DisplayProof
      \end{equation*}
      Note that this would work with any type, as there are no constraints.
      \item \(\lstinline|x| + 1, \aint\). Derivation, assume \lstinline|x| is an
      integer:
      \begin{equation*}
        \AxiomC{\((\lstinline|x|, \aint) \in \{(\lstinline|x|, \aint)\}\)}
        \UnaryInfC{\(\{(\lstinline|x|, \aint)\} \vdash \lstinline|x: Int|\)}
        \AxiomC{\(1 \in \naturals\)}
        \UnaryInfC{\(\{(\lstinline|x|, \aint)\} \vdash \lstinline|1: Int|\)}
        \BinaryInfC{\(\{(\lstinline|x|, \aint)\} \vdash \lstinline|x + 1: Int|\)}
        \DisplayProof
      \end{equation*}
      Due to addition constraining the type of \lstinline|x|, other possible
      types would not work.
      \item \(\lstinline|(x && y)  ==  (x <=  0)|, \abool\). Not well-typed.
      From the left-hand side, we would enforce that \lstinline|x: Bool|, but on
      the right, we find \lstinline|x: Int|. Due to this conflict, there is no
      valid derivation for this term.
      \item \lstinline|f => x => y => f((x, y))|: this is the currying function.
      Note that it will conform to \lstinline|((a, b) => c) => a => b => c| for any choice 
      of \lstinline|a|, \lstinline|b|, and \lstinline|c|. (check)
      \item \lstinline|Cons(x, x)| : \lstinline|List[List[Int]]|. Not
      well-typed. The \(cons\) rule tells us that the second argument must have
      the same type as the result, so \lstinline|x: List[List[Int]]|, but the first
      argument enforces the type to be \lstinline|List[Int]| (again, due to result
      type). As \(\aint \neq \alist{\aint}\), this is not well-typed.

      Note that the singular assignment of \lstinline|x| to \lstinline|Nil()|
      can make a well typed term here, but the typing must hold for \emph{all}
      possible values of \lstinline|x|.
    \end{enumerate}
  \end{solution}
  
\end{exercise}
