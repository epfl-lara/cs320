
% we want lists, pairs
% both parametric
% we want some functions, as well as function definitions
% we want to be able to define and typecheck recursive functions

Consider a type system for a simple functional language, consisting of integers,
booleans, parametric pairs, and lists. The rest of the exercises will revolve
around this system.

% language components
\newcommand{\aint}{\lstinline|int|}
\newcommand{\abool}{\lstinline|bool|}
\newcommand{\aif}{\lstinline|if|}
\newcommand{\athen}{\lstinline|then|}
\newcommand{\aelse}{\lstinline|else|}
\newcommand{\apair}[2]{(#1, #2)}
\newcommand{\afst}[1]{\lstinline|fst|(#1)}
\newcommand{\asnd}[1]{\lstinline|snd|(#1)}
\newcommand{\alist}[1]{\lstinline|List[|#1\lstinline|]|}
\newcommand{\anil}{\lstinline|Nil()|}
\newcommand{\acons}[2]{\lstinline|Cons(|#1, #2\lstinline|)|}
\newcommand{\ato}{\lstinline|=>|}

%%% Actual type system
\begin{gather*}
  % variables
  \AxiomC{\((x, \tau) \in \Gamma\)}
  \RightLabel{(var)}
  \UnaryInfC{\(\Gamma \vdash x : \tau\)}
  \DisplayProof \\
  % integers
  \AxiomC{\(n\) is an integer value}
  \RightLabel{(int)}
  \UnaryInfC{\(\Gamma \vdash \num(n) : \aint\)}
  \DisplayProof \\
  % + -
  \AxiomC{\(e_1 : \aint\)}
  \AxiomC{\(e_2 : \aint\)}
  \RightLabel{(+)}
  \BinaryInfC{\(\Gamma \vdash e_1 + e_2 : \aint\)}
  \DisplayProof \qquad
  \AxiomC{\(e_1 : \aint\)}
  \AxiomC{\(e_2 : \aint\)}
  \RightLabel{(-)}
  \BinaryInfC{\(\Gamma \vdash e_1 - e_2 : \aint\)}
  \DisplayProof \\
  % booleans
  \AxiomC{\(b\) is a boolean value}
  \RightLabel{(bool)}
  \UnaryInfC{\(\Gamma \vdash \abool(b) : \abool\)}
  \DisplayProof \\
  % props
  \AxiomC{\(\Gamma \vdash e_1 : \abool\)}
  \AxiomC{\(\Gamma \vdash e_2 : \abool\)}
  \RightLabel{(and)}
  \BinaryInfC{\(\Gamma \vdash e_1 \land e_2 : \abool\)}
  \DisplayProof \qquad
  \AxiomC{\(\Gamma \vdash e_1 : \abool\)}
  \AxiomC{\(\Gamma \vdash e_2 : \abool\)}
  \RightLabel{(or)}
  \BinaryInfC{\(\Gamma \vdash e_1 \lor e_2 : \abool\)}
  \DisplayProof \\
  \AxiomC{\(\Gamma \vdash e_1 : \abool\)}
  \RightLabel{(not)}
  \UnaryInfC{\(\Gamma \vdash \neg e_1 : \abool\)}
  \DisplayProof \\
  % ite
  \AxiomC{\(\Gamma \vdash e_1 : \abool\)}
  \AxiomC{\(\Gamma \vdash e_2 : \tau\)}
  \AxiomC{\(\Gamma \vdash e_3 : \tau\)}
  \RightLabel{(ite)}
  \TrinaryInfC{\(\Gamma \vdash \aif \, e_1 \, \athen \, e_2 \, \aelse \, e_3 : \tau\)}
  \DisplayProof \\
  % pairs
  \AxiomC{\(\Gamma \vdash e_1 : \tau_1\)}
  \AxiomC{\(\Gamma \vdash e_2 : \tau_2\)}
  \RightLabel{(pair)}
  \BinaryInfC{\(\Gamma \vdash \apair{e_1}{e_2} : \apair{\tau_1}{\tau_2}\)}
  \DisplayProof \\
  % fst
  \AxiomC{\(\Gamma \vdash e : \apair{\tau_1}{\tau_2}\)}
  \RightLabel{(fst)}
  \UnaryInfC{\(\Gamma \vdash \afst{e} : \tau_1\)}
  \DisplayProof \qquad
  % snd
  \AxiomC{\(\Gamma \vdash e : \apair{\tau_1}{\tau_2}\)}
  \RightLabel{(snd)}
  \UnaryInfC{\(\Gamma \vdash \asnd{e} : \tau_2\)}
  \DisplayProof \\
  % lists
  \AxiomC{\phantom{\(e \in \alist{\tau}\)}}
  \RightLabel{(nil)}
  \UnaryInfC{\(\Gamma \vdash \anil : \alist{\tau}\)}
  \DisplayProof \qquad
  \AxiomC{\(\Gamma \vdash e_1 : \tau\)}
  \AxiomC{\(\Gamma \vdash e_2 : \alist{\tau}\)}
  \RightLabel{(cons)}
  \BinaryInfC{\(\Gamma \vdash \acons{e_1}{e_2} : \alist{\tau}\)}
  \DisplayProof \\
  % functions
  \AxiomC{\(\Gamma, x : \tau_1 \vdash e : \tau_2\)}
  \RightLabel{(fun)}
  \UnaryInfC{\(\Gamma \vdash \lambda x : \tau_1 ~\ato~ e : \tau_1 ~\ato~ \tau_2\)}
  \DisplayProof \qquad
  % application
  \AxiomC{\(\Gamma \vdash e_1 : \tau_1 ~\ato~ \tau_2\)}
  \AxiomC{\(\Gamma \vdash e_2 : \tau_1\)}
  \RightLabel{(app)}
  \BinaryInfC{\(\Gamma \vdash e_1 \, e_2 : \tau_2\)}
  \DisplayProof \\
\end{gather*}
