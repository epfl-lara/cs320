
\begin{exercise}{2022}
  Complete (on the next page) the type derivation for the body of the function
  \lstinline|f|.

  \begin{lstlisting}
    def f(x: Int, u: Int, v: Int): Int = {
      if (x < u) {
        u
      }
      else if (v < x) {
        v
      }
      else {
        x
      }
    }
  \end{lstlisting}

  You may use the following type rules:
  %%
  \addtolength{\jot}{2em}
  \begin{gather*}
    \AxiomC{\((x, T) \in \Gamma\)}
    \UnaryInfC{\(\Gamma \vdash x : T\)}
    \DisplayProof 
    \qquad
    % +
    \AxiomC{\(\Gamma \vdash e_1 : Int\)}
    \AxiomC{\(\Gamma \vdash e_2 : Int\)}
    \BinaryInfC{\(\Gamma \vdash e_1 + e_2 : Int\)}
    \DisplayProof
    \quad
    % *
    \AxiomC{\(\Gamma \vdash e_1 : Int\)}
    \AxiomC{\(\Gamma \vdash e_2 : Int\)}
    \BinaryInfC{\(\Gamma \vdash e_1 * e_2 : Int\)}
    \DisplayProof
    \\
    % &&
    \AxiomC{\(\Gamma \vdash e_1 : Bool\)}
    \AxiomC{\(\Gamma \vdash e_2 : Bool\)}
    \BinaryInfC{\(\Gamma \vdash e_1 \&\& e_2 : Bool\)}
    \DisplayProof
    \qquad
    % ||
    \AxiomC{\(\Gamma \vdash e_1 : Bool\)}
    \AxiomC{\(\Gamma \vdash e_2 : Bool\)}
    \BinaryInfC{\(\Gamma \vdash e_1 || e_2 : Bool\)}
    \DisplayProof
    \\
    % lt
    \AxiomC{\(\Gamma \vdash e_1 : Int\)}
    \AxiomC{\(\Gamma \vdash e_2 : Int\)}
    \BinaryInfC{\(\Gamma \vdash e_1 < e_2 : Bool\)}
    \DisplayProof
    \qquad
    % if
    \AxiomC{\(\Gamma \vdash e_1 : Bool\)}
    \AxiomC{\(\Gamma \vdash e_2 : T\)}
    \AxiomC{\(\Gamma \vdash e_3 : T\)}
    \TrinaryInfC{\(\Gamma \vdash if~e_1~then~e_2~else~e_3 : T\)}
    \DisplayProof
  \end{gather*}

  \pagebreak
  \begin{center}
    \scalebox{1.2}{\rotatebox{90}{
      %
        \AxiomC{\((x, Int) \in \Gamma\)}
        \UnaryInfC{\(\Gamma \vdash x : Int\)}
        \AxiomC{\((u, Int) \in \Gamma\)}
        \UnaryInfC{\(\Gamma \vdash u : Int\)}
      \BinaryInfC{\(\Gamma \vdash x < u : Bool\)}
      %
        \AxiomC{\((u, Int) \in \Gamma\)}
      \UnaryInfC{\(\Gamma \vdash u : Int\)}
      %
          \AxiomC{\((\ifAns{v}, \ifAns{Int}) \in \Gamma\)}
          \UnaryInfC{\(\Gamma \vdash \ifAns{v} : \ifAns{Int}\)}
          \AxiomC{\((\ifAns{x}, \ifAns{Int}) \in \Gamma\)}
          \UnaryInfC{\(\Gamma \vdash \ifAns{x} : \ifAns{Int}\)}
        \BinaryInfC{\(\Gamma \vdash \ifAns{v < x} : \ifAns{Bool}\)}
        %
          \AxiomC{\((\ifAns{v}, \ifAns{Int}) \in \Gamma\)}
        \UnaryInfC{\(\Gamma \vdash \ifAns{v} : \ifAns{Int}\)}
        %
          \AxiomC{\((\ifAns{x}, \ifAns{Int}) \in \Gamma\)}
        \UnaryInfC{\(\Gamma \vdash \ifAns{x} : \ifAns{Int}\)}
        %
      \TrinaryInfC{\(\Gamma \vdash \ifAns{if ~ (v < x) ~ v ~ else ~ x}~ : \ifAns{Int}\)}
      %
      \TrinaryInfC{\(\Gamma \vdash if~ (x < u) ~ then~ u ~else ~ if ~ (v < x) ~ v ~ else ~ x ~: Int\)}
      \DisplayProof
    }}
  \end{center}
\end{exercise}

\begin{exercise}{2022, contd}
  For which of the following expressions does type unification succeed? For the
  \(+\) operator, assume the type rules as in the previous question.
  %
  \begin{enumerate}
    \item \(x \Rightarrow y \Rightarrow y(z \Rightarrow 6) + y(7)\)
    \item \(g \Rightarrow f \Rightarrow x \Rightarrow g(f(x))\)
    \item \(x \Rightarrow y \Rightarrow ((z \Rightarrow y), y)\)
    \item \(g \Rightarrow f \Rightarrow x \Rightarrow g(f(x)) + f(g(x)) + x\)
  \end{enumerate}
\end{exercise}

\begin{exercise}{2022, contd}
  Consider a programming language with pairs and the usual typing rules, as in
  the lecture. Apply the unification algorithm on the following function:
  %
  \begin{lstlisting}
    def swap(t) = {
      (t._2, t._1)
    }
  \end{lstlisting}
  %%
  assuming the following type variables assigned to tree nodes:
  %
  \begin{equation*}
    ((t: \tau)._2: \tau_1, (t: \tau)._1: \tau_2): \tau_3
  \end{equation*}
  %%
  Write each step of the unification algorithm, mentioning what rules of the
  algorithm you are applying. We provide you with the initial step:
  %
  \begin{gather*}
    \tau = (\tau_{10}, \tau_1) \\
    \tau = (\tau_{2}, \tau_{20}) \\
    \tau_3 = (\tau_{1}, \tau_{2})
  \end{gather*}

  \begin{solution}
    \begin{enumerate}
      \item Substituting \(\tau = (\tau_{10}, \tau_1)\):
      \begin{gather*}
        (\tau_{10}, \tau_1) = (\tau_{2}, \tau_{20}) \\
        \tau_3 = (\tau_{1}, \tau_{2})
      \end{gather*}
      \item Unifying the pair expression:
      \begin{gather*}
        \tau_{10} = \tau_{2} \\
        \tau_1 = \tau_{20} \\
        \tau_3 = (\tau_{1}, \tau_{2})
      \end{gather*}
      \item Substituting \(\tau_1\) and \(\tau_2\) per equations:
      \begin{gather*}
        \tau_3 = (\tau_{1}, \tau_{2}) 
      \end{gather*}
    \end{enumerate}

    We can get the values of \(\tau\) and \(\tau_3\) in terms of \(\tau_1\) and
    \(\tau_2\) by looking at intermediate steps we took during unification.
  \end{solution}

  Write down an expression for the argument and return types of \lstinline|swap|
  in terms of the type variables \(\tau_1\) and \(\tau_2\).

  \begin{solution}
    Argument type: \(\tau = (\tau_2, \tau_1)\) \\

    Return type: \(\tau_3 = (\tau_1, \tau_2)\)
  \end{solution}
\end{exercise}
