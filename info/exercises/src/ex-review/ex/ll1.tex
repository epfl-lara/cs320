
\begin{exercise}{2016}
  Consider the grammar:
  \begin{align*}
    decl &::= varDecl \mid funDecl \\
    varDecl &::= type ~\lstinline|ID|; \\
    funDecl &::= type ~\lstinline|ID|~ ( optIDs ); \\
    optIDs &::= \epsilon \mid IDs \\
    IDs &::= \lstinline|ID| \mid IDs ~\lstinline|ID| \\
    type &::= \lstinline|int| \mid type \lstinline|*| \\
  \end{align*}

  \begin{enumerate}
    \item Compute \(\nullable\) and \(\first\) for each non-terminal of the grammar above.
    \begin{solution}
      Only \(optIDs\) is nullable. The first sets are:
      %
      \begin{gather*}
        \first(decl) = \{\lstinline|int|\} \\
        \first(varDecl) = \{\lstinline|int|\} \\
        \first(funDecl) = \{\lstinline|int|\} \\
        \first(optIDs) = \{\lstinline|ID|\} \\
        \first(IDs) = \{\lstinline|ID|\} \\
        \first(type) = \{\lstinline|int|\}
      \end{gather*}
    \end{solution}
    \item Explain why the grammar is not LL(1).
    \begin{solution}
      \(decl\), \(IDs\), and \(type\) each have two rules with the same
      \(\first\) set. 
    \end{solution}
    \item Give an LL(1) grammar describing the same sequences of tokens as the
    previous grammar.
    \begin{solution}
      We can remove the common prefix from \(decl\):
      \begin{gather*}
        decl = type~\lstinline|ID|~decl' \\
        decl' = \lstinline|;| \mid \lstinline|(| ~optIDs~\lstinline|);|
      \end{gather*}

      We can remove the left recursion from the \(IDs\) rule:
      \begin{gather*}
        IDs = \lstinline|ID| ~IDs' \\
        IDs' = \epsilon \mid \lstinline|ID| ~IDs'
      \end{gather*}

      We can also remove the left recursion from the \(type\) rule:
      \begin{gather*}
        type = \lstinline|int| ~type' \\
        type' = \epsilon \mid \lstinline|*| ~type'
      \end{gather*}

      Note that \(IDs\) can be followed only by a closing parenthesis.
    \end{solution}
  \end{enumerate}
\end{exercise}

\begin{exercise}{2022}
  Consider the following grammar with non-terminals \(S\) and \(A\) and
  terminals \(\eof, (, ), [, \text{ and }]\):

  \begin{align*}
    S &::= A ~\eof \\
    A &::= (A) ~A \mid A~[A] \mid \epsilon
  \end{align*}

  \begin{enumerate}
    \item Choose all true statements about the grammar above:
    \begin{enumerate}
      \item \(``[]()([)]"\) is accepted by the grammar.
      \item The grammar is LL(1).
      \item The grammar is ambiguous. \ifAns{\checkmark}
      \item \(\nullable(A) = true\). \ifAns{\checkmark}
      \item \(\nullable(S) = true\).
    \end{enumerate}
    \item Choose the correct option:
    \begin{enumerate}
      \item \(\first(S) = \{\eof\}\)
      \item \(\first(S) = \{(, [\}\)
      \item \(\first(S) = \{(, ), \eof\}\)
      \item \(\first(S) = \{(, [, \eof\}\) \ifAns{\checkmark}
      \item \(\first(S) = \{(,), [,], \eof\}\)
    \end{enumerate}
    \item Choose the correct option:
    \begin{enumerate}
      \item \(\follow(A) = \{), ]\}\)
      \item \(\follow(A) = \{), ], \eof\}\)
      \item \(\follow(A) = \{(, [, ), ]\}\)
      \item \(\follow(A) = \{(, [, ], \eof\}\)
      \item \(\follow(A) = \{(, [, ), ], \eof\}\)
    \end{enumerate}
    \ifAns{None of the above are correct; \(\follow(A) = \{), [, ], \eof\}\)}
  \end{enumerate}
\end{exercise}
