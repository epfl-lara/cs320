
\begin{exercise}{}

  For each of the following languages, give a context-free grammar that
  generates it:

  \begin{enumerate}
    \item \(L_1 = \{a^nb^m \mid n, m \in \naturals \land n \geq 0 \land m \geq n\}\)
    \item \(L_2 = \{a^nb^mc^{n+m} \mid n, m \in \naturals\}\)
    \item \(L_3 = \{w \in \{a, b\}^* \mid \exists m \in \naturals.\; |w| = 2m +
    1 \land w_{(m+1)} = a \}\) (\(w\) is of odd length, has \(a\) in the middle)
  \end{enumerate}

  \begin{solution}
    \begin{enumerate}
      \item \(L_1 = \{a^nb^m \mid n, m \in \naturals \land n \geq 0 \land m \geq n\}\)
      \begin{align*}
        S &::= aSb \mid B\\
        B &::= bB \mid \epsilon
      \end{align*}
      \item \(L_2 = \{a^nb^mc^{n+m} \mid n, m \in \naturals\}\)
      \begin{align*}
        S &::= aSc \mid B\\
        B &::= bBc \mid \epsilon
      \end{align*}

      A small tweak to \(L_1\)'s grammar allows us to keep track of addition
      precisely here. Could we do something similar for \(\{a^nb^nc^n \mid n \in
      \naturals\}\)? (open-ended discussion)

      \item \(L_3 = \{w \in \{a, b\}^* \mid \exists m \in \naturals.\; |w| = 2m +
      1 \land w_{(m+1)} = a \}\)
      \begin{align*}
        S &::= aSb \mid bSa \mid aSa \mid bSb \mid a
      \end{align*}

      Note that after each recursive step, the length of the inner string has
      the same parity (i.e. odd).
    \end{enumerate}    
  \end{solution}
  
\end{exercise}

\begin{exercise}{}

  Consider the following context-free grammar \(G\):

  \begin{align*}
    A &::= -A \\
    A &::= A - \textit{id} \\
    A &::= \textit{id} \\
  \end{align*}

  \begin{enumerate}
    \item Show that \(G\) is ambiguous, i.e., there is a string that has two
    different possible parse trees with respect to \(G\).
    \item Make two different unambiguous grammars recognizing the same words,
    \(G_p\), where prefix-minus binds more tightly, and \(G_i\), where
    infix-minus binds more tightly.
    \item Show the parse trees for the string you produced in (1) with respect
    to \(G_p\) and \(G_i\).
    \item Produce a regular expression that recognizes the same language as
    \(G\).
  \end{enumerate}

  \begin{solution}
    \begin{enumerate}
      \item An example string is \(- \textit{id} - \textit{id}\). It can be
      parsed as either \(-(\textit{id} - \textit{id})\) or \((- \textit{id}) -
      \textit{id}\). The corresponding parse trees are:

      \begin{center}
        \begin{forest}
          [\(A\)
            [\(A\)
              [\(-\)]
              [\(\textit{id}\)]
            ]
            [\(-\)]
            [\(\textit{id}\)]
          ]
        \end{forest}
        \hspace{10ex}
        \begin{forest}
          [\(A\)
            [\(-\)]
            [\(A\)
              [\(A\)
                [\(\textit{id}\)]
              ]
              [\(-\)]
              [\(\textit{id}\)]
            ]
          ]
        \end{forest}
      \end{center}

      Left: prefix binds tighter, right: infix binds tighter.

      \item \(G_p\):
      \begin{align*}
          A &::= B \mid A - \textit{id} \\
          B &::= -B \mid \textit{id}
      \end{align*}

      \(G_i\):
      \begin{align*}
          A &::= C \mid -A \\
          C &::= \textit{id} \mid C - \textit{id}
      \end{align*}

      \item Parse trees for \(- \textit{id} - \textit{id}\) with respect to \(G_p\) (left)
      and \(G_i\) (right):

      \begin{center}
        \begin{forest}
          [\(A\)
            [\(A\)
              [\(B\)
                [\(-\)]
                [\(B\)
                  [\(\textit{id}\)]
                ]
              ]
            ]
            [\(-\)]
            [\(\textit{id}\)]
          ]
        \end{forest}
        \hspace{10ex}
        \begin{forest}
          [\(A\)
            [\(-\)]
            [\(A\)
              [\(C\)
                [\(\textit{id}\)]
              ]
              [\(-\)]
              [\(\textit{id}\)]
            ]
          ]
        \end{forest}
      \end{center}

      \item \(L(G) = L(-^*\textit{id} (-\textit{id})^*)\). Note: \(()\) are part
      of the regular expression syntax, not parentheses in the string.

    \end{enumerate}
  \end{solution}

\end{exercise}


\begin{exercise}{}

  Consider the two following grammars \(G_1\) and \(G_2\):

  \begin{align*}
    G_1: & \\
    S &::= S(S)S \mid \epsilon \\
    G_2: & \\
    R &::= RR \mid (R) \mid \epsilon
  \end{align*}

  \noindent
  Prove that:
  \begin{enumerate}
    \item \(L(G_1) \subseteq L(G_2)\), by showing that for every parse tree in
    \(G_1\), there exists a parse tree yielding the same word in \(G_2\).
    \item (Bonus) \(L(G_2) \subseteq L(G_1)\), by showing that there exist
    equivalent parse trees or derivations.
  \end{enumerate}

  \begin{solution}

    \begin{enumerate}
      \item \(L(G_1) \subseteq L(G_2)\).
      
      We give a recursive transformation of parse trees in \(G_1\) producing
      parse trees in \(G_2\). 

      \begin{enumerate}
        \item \textbf{Base case:} The smallest parse tree is the \(\epsilon\)
        production, which can be transformed as (left to right):
        \begin{center}
          \begin{forest}
            [\(S\)
              [\(\epsilon\)]
            ]
          \end{forest}
          \hspace{8ex}
          \begin{forest}
            [\(R\)
              [\(\epsilon\)]
            ]
          \end{forest}
        \end{center}
        \item \textbf{Recursive case:} Rule \(S ::= S(S)S\). The parse tree transformation is:
        \begin{center}
          \begin{forest}
            [\(S\)
              [\(S_1\)]
              [\((_2\)]
              [\(S_3\)]
              [\()_4\)]
              [\(S_5\)]
            ]
          \end{forest}
          \hspace{10ex}
          \begin{forest}
            [\(R\)
              [\(R_1\)]
              [\(R\)
                [\(R\)
                  [\((_2\)]
                  [\(R_3\)]
                  [\()_4\)]
                ]
                [\(R_5\)]
              ]
            ]
          \end{forest}
        \end{center} 

        The nodes are numbered to check that the order of children (left to
        right) does not change. This ensures that the word yielded by the tree
        is the same. The transformation is applied recursively to the children
        \(S_1, S_3, S_5\) to obtain \(R_1, R_3, R_5\).

        Verify that the tree on the right is indeed a parse tree in \(G_2\).
      \end{enumerate}

      \item \(L(G_2) \subseteq L(G_1)\). 
      
      Straightforward induction on parse trees does not work easily. The rule
      \(R ::= RR\) in \(G_2\) is not directly expressible in \(G_1\) by a simple
      transformation of parse trees. However, we can note that, in fact, adding
      this rule to \(G_1\) does not change the language!

      Consider the grammar \(G_1'\) defined by \(S ::= SS \mid S(S)S \mid
      \epsilon\). We must show that for every two words \(v\) and \(w\) in
      \(L(G_1)\), \(vw\) is in \(L(G_1)\), and so adding the rule \(S ::= SS\)
      does not change the language.

      We induct on the length \(|v| + |w|\). 
      
      \begin{enumerate}
        \item \textbf{Base case:} \(|v| + |w| = 0\). \(v = w = vw = \epsilon \in
        L(G_1)\). QED.
        \item \textbf{Inductive case:} \(|v| + |w| = n + 1\). The induction
        hypothesis is that for every \(v', w'\) with \(|v'| + |w'| = n\), \(v'w'
        \in L(G_1)\).

        From the grammar, we know that either \(v = \epsilon\) or \(v = x(y)z\)
        for \(x, y, z \in L(G_1)\). If \(v = \epsilon\), then \(w = vw \in
        L(G_1)\). In the second case, \(vw = x(y)zw\). However, \(zw \in
        L(G_1)\) by the inductive hypothesis, as \(|z| + |w| < n \).

        Thus, \(vw = x(y)z'\) for \(z' \in L(G_1)\). Finally, since \(x, y, z'
        \in L(G_1)\), it follows from the grammar rules that \(vw = x(y)z' \in
        L(G_1)\). 
      \end{enumerate}
      
      Thus, \(L(G_1) = L(G_1')\). It can now be shown just as in the first part,
      that \(L(G_2) \subseteq L(G_1')\).
    \end{enumerate}
    
  \end{solution}
  
\end{exercise}

\begin{exercise}{}
  
  Consider a context-free grammar \(G = (A, N, S, R)\). Define the reversed
  grammar \(rev(G) = (A, N, S, rev(R))\), where \(rev(R)\) is the set of rules
  is produced from \(R\) by reversing the right-hand side of each rule, i.e.,
  for each rule \(n ::= p_1 \ldots p_n\) in \(R\), there is a rule \(n ::=
  p_n \ldots p_1\) in \(rev(R)\), and vice versa. The terminals,
  non-terminals, and start symbol of the language remain the same.

  For example, \(S ::= abS \mid \epsilon\) becomes \(S ::= Sba \mid \epsilon\).

  Is it the case that for every context-free grammar \(G\) defining a language
  \(L\), the language defined by \(rev(G)\) is the same as the language of
  reversed strings of \(L\), \(rev(L) = \{rev(w) \mid w \in L\}\)? Give a proof
  or a counterexample.

  \begin{solution}

    Consider any word \(w\) in the original language. Looking at the definition
    of a language \(L(G)\) defined by a grammar \(G\):
    \begin{equation*}
      w \in L(G) \iff \exists T.\; w = yield(T) \land isParseTree(G, T) 
    \end{equation*}

    There must exist a parse tree \(T\) for \(w\) with respect to \(G\). We must
    show that there exists a parse tree for \(rev(w)\) with respect to the
    reversed grammar \(G_r = rev(G)\) as well.

    We propose that this is precisely the tree \(T_r = mirror(T)\). Thus, we
    need to show that \(rev(w) = yield(T_r)\) and that \(isParseTree(G_r,
    T_r)\).

    \begin{enumerate}
      \item \(rev(w) = yield(T_r)\): \(yield(\cdot)\) of a tree is the word
      obtained by reading its leaves from left to right. Thus, the yield of the
      mirror of a tree \(yield(mirror(\cdot))\) is the word obtained by reading
      the leaves of the original tree from right to left. Thus, \(yield(T_r) =
      yield(mirror(T)) = rev(yield(T)) = rev(w)\).

      \item \(isParseTree(G_r, T_r)\): We need to show that \(T_r\) is a parse
      tree with respect to \(G_r\). Consider the definition of a parse tree:
      \begin{enumerate}
        \item The root of \(T_r\) is the start symbol of \(G_r\): the root of
        \(T_r = mirror(T)\) is the same as that of \(T\). Since \(T\)'s root
        node must be the start symbol of \(G\), it is also the root symbol of
        \(T_r\). \(G\) and \(G_r\) share the same start symbol in our
        transformation.
        \item The leaves are labelled by the elements of \(A\): the mirror
        transformation does not alter the set or the label of leaves, only their
        order. This property transfers from \(T\) to \(T_r\) as well.
        \item Each non-leaf node is labelled by a non-terminal symbol: the
        mirror transformation does not alter the label of non-leaf nodes either,
        so this property transfers from \(T\) to \(T_r\) as well.
        \item If a non-leaf node has children that are labelled \(p_1, \ldots,
        p_n\) left-to-right, then there is a rule \((n ::= p_1 \ldots p_n)\) in
        the grammar: consider any non-leaf node in \(T_r\), labelled \(n\), with
        children labelled left-to-right \(p_1, \ldots, p_n\). By the definition
        of \(mirror\), the original tree \(T\) must have the same node labelled
        \(n\), with the reversed list of children left-to-right, \(p_n, \ldots,
        p_1\). Since \(T\) is a parse tree for \(G\), \(n ::= p_n \ldots p_1\)
        is a valid rule in \(G\), and by the reverse transformation, \(n ::= p_1
        \ldots p_n\) must be a rule in \(G_r\). Thus, the property is satisfied.
      \end{enumerate}
    \end{enumerate}

    Thus, both properties are satisfied. Therefore, the language defined by the
    reversed grammar is the reversed language of the original grammar.

  \end{solution}

\end{exercise}

